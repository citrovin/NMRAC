\subsection{Literature Review}
\label{sec:related-work}
Model reference adaptive control (MRAC) was first introduced in the early 1970s by Whitaker et al. \cite{whitakerDesignModelReference1958}. Originally, it was thought to deal with process uncertainties and disturbance dynamics, however, the proposed techniques were also used in different contexts, including but not limited to auto-tuning, automatic construction of gain schedules, and adaptive filtering \cite{astromTheoryApplicationsAdaptive1983, astromHistoryAdaptiveControl2014}.

The adaptation mechanism of parameters follows two main approaches, namely 1) the MIT method or gradient descent-based method, and 2) the Lyapunov-based method \cite{astromAdaptiveControl2008}. The MIT method does not come with stability guarantees \cite{mareelsRevisitingMitRule1987}, whereas the Lyapunov method is based on an adaptation rule derived from Lyapunov's second method \cite{shackclothSynthesisModelReference1965}. It imposes stability since the adaptation rule is chosen in a way such that the decrease condition on the Lyapunov function is always satisfied, thus, implying system convergence.

% 1
% Generally, MRACs can be split into three categories, firstly \textit{direct} and secondly \textit{indirect MRAC}. The former aims to adapt controller parameters directly and the latter aims to update the model parameters. The third category is a hybrid approach called \textit{Combined/Composite MRAC} (CMRAC), first proposed by Duarte et al. in \cite{duarteCombinedDirectIndirect1989} for first-order systems. The approach is based on estimating two parts with two separate adaptation rules, namely 1) unmatched model uncertainty and 2) system parameters. Their approach showed in simulations to be more robust than both direct and indirect adaptive control if considered individually \cite{narendraRobustAdaptiveControl1988}, however, this has yet to be proven. CMRACs are generalizable for $n$-th order linear systems, as proposed by Lavretsky \cite{lavretskyRobustAdaptiveControl2013} and Tao et al. \cite{taoAdaptiveControlSystems2013}. Additionally, Lavretsky proposes using radial basis function (RBF) NNs to perform system identification and estimate the unmatched model uncertainties for MRACs, with proven stability guarantees, which enables the application of this method to a larger class of nonlinear systems \cite{lavretskyCombinedCompositeModel2009}.

% 2.
Generally, MRACs can be categorized into the following three categories \cite{astromHistoryAdaptiveControl2014}:
\begin{enumerate}
    \item \textit{direct MRAC} updates the controller parameters directly, ensuring the closed-loop system behaves like the reference model,
    \item \textit{indirect MRAC} updates an estimated model of the plant, which is then used to compute the controller parameters. Unlike direct MRAC, where the controller is updated directly, indirect MRAC continuously refines the plant model and adapts the control law accordingly, and
    \item \textit{Combined MRAC} (CMRAC), introduced by Duarte et al. \cite{duarteCombinedDirectIndirect1989}, integrates both methods using two adaptation rules: (1) estimating unmatched model uncertainties and (2) identifying system parameters. Simulations suggest that CMRAC is more robust than either direct or indirect MRAC alone \cite{narendraRobustAdaptiveControl1988}, though formal guarantees are still an open question.
\end{enumerate}

CMRACs have been extended to $n$-th order linear systems \cite{lavretskyRobustAdaptiveControl2013, taoAdaptiveControlSystems2013}. Additionally, Lavretsky \cite{lavretskyCombinedCompositeModel2009} incorporated Radial Basis Function (RBF) NNs for system identification, enabling unmatched uncertainty estimation with stability guarantees, broadening its applicability to a larger class of nonlinear systems.

% 3.
% Generally, MRACs can be categorized into three types: direct, \textit{indirect}, and \textit{combined MRACs}. In direct MRAC, the controller parameters are adapted directly, whereas in indirect MRAC, the model parameters are updated first, and the controller is adjusted accordingly.

% MRACs can be categorized into three types: \textit{direct}, \textit{indirect}, and \textit{combined MRACs}. In direct MRAC, controller parameters are adapted directly, whereas in indirect MRAC, model parameters are updated first.

% Combined (or Composite) MRAC, introduced by Duarte et al. \cite{duarteCombinedDirectIndirect1989}, integrates both approaches using two adaptation rules: (1) estimating unmatched model uncertainties and (2) identifying system parameters. Simulations suggest greater robustness compared to direct or indirect MRAC alone \cite{narendraRobustAdaptiveControl1988}, though formal guarantees remain an open question.

% CMRAC has been extended to n-th order linear systems \cite{lavretskyRobustAdaptiveControl2013, taoAdaptiveControlSystems2013}. Lavretsky further incorporated Radial Basis Function (RBF) Neural Networks for system identification, enabling unmatched uncertainty estimation with proven stability guarantees, broadening its applicability to nonlinear systems \cite{lavretskyCombinedCompositeModel2009}.

The proposed algorithm in this paper is a direct MRAC method, utilizing a Lyapunov-based learning mechanism, that includes nonlinearities in the form of a simple feedforward NN, with a nonlinear activation function.

\subsection{Contributions}
\label{sec:contributions}
The method proposed in this work is based on the logic of the previously mentioned works and the novelty of this approach lies in the ability of learning a stabilizing controller, in spite of a nonlinear NN being present in the closed-loop system. Hence, this method is called \textit{nonlinear MRAC} or \textit{NMRAC}. Additionally, the proposed algorithm is validated in simulations, which show the convergence of the NN parameters to their desired values.