\section{Discussion}
\label{sec:discussion}
The simulation results show that the proposed update law enables the system to learn the desired parameters across different reference signals. The speed of convergence is influenced by the learning rate $c$, where smaller values accelerate convergence while larger values slow it down. However, a limitation of the method is its sensitivity to numerical instabilities, which may not be immediately apparent from the presented simulations. These instabilities can be mitigated by reducing the sampling time, thereby improving the precision of integral and derivative approximations and enhancing overall simulation accuracy.

\section{Conclusion}
\label{sec:conclusion}
This paper presents a novel algorithm for learning NNCs with guaranteed stability properties. The proposed update law builds upon the theory of linear MRAC and extends it to handle nonlinearities within the controller. Simulation results demonstrate that the algorithm yields the expected results, leading the controlled system to converge towards the model reference system with stability guarantees. While numerical instabilities pose a limitation, appropriate tuning of the sampling time can mitigate their impact. These findings highlight the potential of the proposed approach for adaptive control applications involving nonlinear systems.

\section{Outlook}
\label{sec:outlook}
Future work should focus on refining the update law to enhance robustness against numerical instabilities, potentially through deriving direct discrete update laws. Additionally, extending the methodology to experimental validation on physical systems would provide deeper insights into its real-world applicability and further validate its effectiveness in practical control scenarios.